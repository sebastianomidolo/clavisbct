% Perché non accada mai più
% Questo documento e' stato generato automaticamente dal software \"SenzaParola\", scritto da Sebastiano Midolo (midolo@multiopac.com) per le Biblioteche Civiche di Torino e rilasciato sotto licenza GPL (http://www.gnu.org/licenses)
\documentclass[a4paper]{article}
\usepackage{fancyhdr,fixmarks}
\usepackage{graphicx}
\usepackage[latin1]{inputenc}
\parindent 0pt
\fancyhead{}
\fancyfoot{}
\newcounter{scheda}
\setcounter{scheda}{1}
\newcommand{\scheda}{\bigskip{\bf\thescheda.\hspace{2ex}}\addtocounter{scheda}{1}}
\newcommand{\inizioscheda}{\par\medskip{\bf\thescheda.\hspace{2ex}}\addtocounter{scheda}{1}}
\newcommand{\mainentry}[1]{{\bf #1\par\nopagebreak}}
\newcommand{\descr}[1]{#1}
\newcommand{\note}[1]{{\par\nopagebreak\hspace{2ex}{\small\sc #1}}}
\newcommand{\collciv}[1]{{\par\nopagebreak\hspace{2ex}{\small\sc Civica centrale: {\rm #1}}}}
\newcommand{\colldec}[2]{{\par\nopagebreak\hspace{2ex}{\small\sc Decentrate: {\rm #1} (#2)}}}
\newcommand{\impronta}[1]{\par\nopagebreak\hspace{2ex}{\small\sc Impronta: \bf #1}}
\input local.tex
\pagestyle{fancy}
\begin{document}
\begin{samepage}
\graphicspath{{/home/midolo/loghi/} {/home/midolo/ProgettiCivica/SenzaParola/Projects/PpV3e55mOMgAADvdENM/Img/}}
\par\includegraphics[scale=.25]{logo1}\hfill\includegraphics[scale=.25]{logo2}\par\vspace{8mm}
{\centerline{\includegraphics[scale=1]{formato_latex-img_pp.jpg}}\par\vspace{8mm}}
\centerline{\huge Perché non accada mai più}
\bigskip\centerline{\large Libri fascisti per la scuola. Il testo unico di Stato (1929-1943)}\par\vspace{5mm}
\bigskip{\sl \hspace{18pt}``... L'adesione del regime fascista al testo unico ebbe ragioni ovvie, trasparenti nella qualità, nel contenuto, nella grafica e nelle immagini. I testi unici erano la via migliore per imbibire maestre e maestri e bimbi e bimbe dell'ideologia del regime... Il testo editorialmente unico d'età fascista cercava proprio questo: la propaganda del regime, sia esplicita sia subliminale, senza contraddizioni``.\par
{\footnotesize Tullio De Mauro}
\par\smallskip
Mostra documentaria dal 14 al 26 aprile 2003 presso la Biblioteca civica A. Geisser, in collaborazione con la Divisione Servizi Educativi della Città di Torino.\par 
{\footnotesize Testi di Tullio De Mauro, Alberto Monticone e Nicola Tranfaglia. Ricerche storiche e iconografiche di Aldo Zambelloni.}\par\smallskip
Le opere di seguito segnalate sono disponibili presso la Biblioteca Civica centrale.}\par\bigskip
% L'ideologia dell'epoca
\section*{\centerline {L'ideologia dell'epoca}}
\inizioscheda
\mainentry{Codignola, Ernesto}
\descr{Il problema dell'educazione nazionale in Italia / Ernesto Codignola.~-- Firenze~: Vallecchi, stampa 1925.~-- 358 p.~; 20 cm.~-- (La nostra scuola~; 30)}
\collciv{188.E.6}
\inizioscheda
\mainentry{Gasperoni, Gaetano}
\descr{Il fascismo nella scuola / Gaetano Gasperoni.~-- Milano~; Verona~: A. Mondadori, 1929.~-- VIII, 202 p., 17 c. di tav.~: ill.~; 22 cm}
\collciv{222.D.5}
\inizioscheda
\mainentry{Giuliano, Balbino}
\descr{La politica scolastica del governo nazionale / Balbino Giuliano.~-- Milano~: Alpes, 1934.~-- 171 p.~; 20 cm}
\collciv{221.C.34}
\inizioscheda
\mainentry{Ignotus}
\descr{Stato fascista, Chiesa e scuola / Ignotus.~-- Roma~: Libreria del littorio, stampa 1929.~-- 99 p.~; 20 cm}
\collciv{221.D.29}
\inizioscheda
\mainentry{Licitra, Carmelo}
\descr{La nuova scuola del popolo italiano / Carmelo Licitra.~-- Roma~: De Alberti, 1924.~-- 173 p.~; 20 cm.~-- (Studi pedagogici~; 2)}
\collciv{188.B.12}
\inizioscheda
\mainentry{Mannocchi, Giuseppe}
\descr{Fascismo e scuola / Giuseppe Mannocchi.~-- Falconara Marittima (AN)~: Sabatini, 1935.~-- 206 p.~; 18 cm}
\collciv{223.F.12}
\inizioscheda
\mainentry{Mazzetti, Roberto}
\descr{Scuola e nazione sul piano dell'impero / Robero Mazzetti.~-- 2. ed.~-- Bologna~: La Diana Scolastica, 1937.~-- VIII, 266 p.~; 21 cm}
\collciv{226.E.41}
\inizioscheda
\mainentry{Mussolini, Benito}
\descr{La dottrina del fascismo / Benito Mussolini~; con commento a cura di Alfredo Giovannetti.~-- Torino~: Paravia, 1937.~-- IV, 63 p.~; 19 cm}
\collciv{239.LD.3}
\inizioscheda
\mainentry{Padellaro, Nazareno}
\descr{Fascismo educatore / Nazareno Padellaro.~-- Roma~: Cremonese, 1938.~-- XII, 234 p.~; 22 cm}
\collciv{224.D.4}
\inizioscheda
\mainentry{Pagliaro, Antonino}
\descr{La scuola fascista / Antonino Pagliaro.~-- Milano~: A. Mondadori, 1939.~-- 100 p.~; 19 cm.~-- (Panorami di vita fascista)}
\collciv{221.E.45}
\inizioscheda
\mainentry{Romanini, Luigi}
\descr{I principi del fascismo nel campo dell'educazione / Luigi Romanini.~-- Torino [etc.]~: Paravia, 1935.~-- 375 p.~; 20 cm}
\collciv{222.F.39}
\inizioscheda
\mainentry{Romanini, Luigi}
\descr{I principi del fascismo nel campo dell'educazione / Luigi Romanini.~-- 2. ed.~-- Torino [etc.]~: Paravia, 1939.~-- 368 p.~; 19 cm.~-- In appendice: La Carta della scuola e la relazione al Duce di S. E. Bottai}
\collciv{224.F.37}
\inizioscheda
\mainentry{Romanini, Luigi}
\descr{Scuola littoria~: fondamento dottrinale corporativo della educazione fascista / Luigi Romanini.~-- Torino [etc.]~: Paravia, 1935.~-- 219 p.~; 20 cm}
\collciv{223.F.14}
\inizioscheda
\mainentry{Roncarà, Dino}
\descr{Saggi sull'educazione fascista / Dino Roncarà.~-- Bologna~: La Diana Scolastica, 1938.~-- XVI, 191 p.~; 19 cm}
\collciv{224.F.24}
\inizioscheda
\descr{La {\bf scuola} fascista / scritti di Mario Alla ... [et al.].~-- Tivoli (ROMA)~: Arti Grafiche Chicca, 1938.~-- 78 p.~; 26 cm.~-- (Collezione di Studi della Rivista ``La Terra``~; 3)}
\collciv{441.LD.39}
\inizioscheda
\mainentry{Terzaghi, Nicola}
\descr{Scuola libera e scuola di Stato~: aspetti economici e possibilità morali / N. Terzaghi.~-- Milano~: Imperia, stampa 1923.~-- 48 p.~; 19 cm.~-- (Collana Imperia~; 7)}
\collciv{411.LF.8}
\inizioscheda
\mainentry{Tesini, Oddone}
\descr{Idealità fasciste nella scuola / Oddone Tesini.~-- Bologna~: Cappelli, 1927.~-- 98 p.~; 19 cm}
\collciv{222.F.6}
\inizioscheda
\mainentry{Volpicelli, Luigi}
\descr{Apologia della scuola media / Luigi Volpicelli.~-- Roma~: Istituto nazionale di cultura fascista, [1941].~-- 63 p.~; 23 cm.~-- (Quaderni di cultura politica. Serie XI~; 5).~-- Suppl. a: Civiltà fascista, n. 10 (ott. 1941)}
\collciv{411.LD.81}

% La storiografia contemporanea
\section*{\centerline {La storiografia contemporanea}}
\inizioscheda
\mainentry{Biondi, Giovanni}
\descr{... Voi siete la primavera d'Italia...~: l'ideologia fascista nel mondo della scuola, 1925-1943 / Giovanni Biondi e Fiora Imberciadori~; postfazione/testimonianza di Lucio Lombardo Radice.~-- Torino~: Paravia, 1982.~-- 220 p., [5] c. di tav.~: ill.~; 23 cm.~-- (Percorsi~; 21).~-- Con una raccolta di testi e documenti}
\collciv{449.D.35}
\inizioscheda
\mainentry{Charnitzky, Jürgen}
\descr{Fascismo e scuola~: la politica scolastica del regime, 1922-1943 / Jürgen Charnitzky.~-- Scandicci (FI)~: La nuova Italia, c1996.~-- XX, 603 p.~; 21 cm.~-- (Biblioteca di storia~; 61).~-- Trad. di Laura Sergo Burge, rev. di Ina Pizzuto.~-- ISBN 882210224X}
\collciv{696.E.20}
\inizioscheda
\mainentry{Gentili, Rino}
\descr{Giuseppe Bottai e la riforma fascista della scuola / Rino Gentili.~-- Firenze~: La nuova Italia, 1979.~-- VI, 218 p.~; 21 cm.~-- (Educatori antichi e moderni~; 347)}
\collciv{612.F.26}
\inizioscheda
\mainentry{Isnenghi, Mario}
\descr{L'educazione dell'Italiano~: il fascismo e l'organizzazione della cultura / Mario Isnenghi.~-- Bologna~: Cappelli, c1979.~-- 471 p.~; 19 cm.~-- (Saggi~; 2)}
\collciv{448.M.11}
\inizioscheda
\mainentry{Mazzatosta, Teresa Maria}
\descr{Il regime fascista tra educazione e propaganda (1935-1943) / Teresa Maria Mazzatosta.~-- Bologna~: Cappelli, c1978.~-- XII, 7-243 p.~; 19 cm.~-- (Nuova universale Cappelli~; 6)}
\collciv{448.E.86}
\inizioscheda
\descr{{\bf Opposizioni} alla riforma Gentile / contributi di Giorgio Chiosso ... [et al.].~-- Torino~: Centro studi Carlo Trabucco, stampa 1985.~-- 187 p.~; 24 cm.~-- (Quaderni del Centro studi Carlo Trabucco~; 7)}
\collciv{285.LD.7,  271.A.43}
\inizioscheda
\mainentry{Ostenc, Michel}
\descr{La scuola italiana durante il fascismo / Michel Ostenc.~-- Roma~; Bari~: Laterza, 1981.~-- 312 p.~; 21 cm.~-- (Storia e società)}
\collciv{449.F.48}
\inizioscheda
\descr{La {\bf piccola} italiana / a cura di Fiorenzo Alfieri ... [et al.].~-- Firenze~: Manzuoli, 1977.~-- 32 p.~: ill.~; 19 cm.~-- (Biblioteca di lavoro)}
\collciv{237.LD.50}
\inizioscheda
\descr{{\bf Scuola} e educazione in Emilia Romagna fra le due guerre / a cura di Aldo Berselli e Vittorio Telmon.~-- Bologna~: CLUEB, stampa 1983.~-- 623 p.~; 22 cm}
\collciv{612.F.426}

\end{samepage}
\end{document}
